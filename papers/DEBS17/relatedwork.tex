\section{Related Work}\label{sec:rw}

%\vs{to do from scratch, should cover: stream processing papers, event-based middlewares, middleware that exploits hardware features}

A few dedicated solutions exist today for distributed stream processing using reactive programming.
For instance, \textsc{Reactive Kafka}~\cite{reactivekafka} allows stream processing atop of Apache \textsc{Kafka}~\cite{apachekafka}.
These solutions do not, however, support secure execution in a trusted execution environment.

More recently, some open-source middleware frameworks (\emph{e.g.}, Apache \textsc{Spark}~\cite{apachesparkstreaming}, Apache \textsc{Storm}~\cite{apachestorm}, \textsc{Infinispan}~\cite{infinispan}) introduced APIs to allow developers to quickly set up and deploy stream processing infrastructures.
These systems rely on the \emph{Java} virtual machine (JVM)~\cite{lindholm2014java}.
However, SGX currently imposes a hard memory limit of 128\,MB to the enclaved code and data, at the cost of expensive encrypted memory paging mechanisms and serious performance overheads~\cite{pires_scbr:2016,brenner_securekeeper:_2016} when this limit is crossed.
Moreover, executing a fully-functional JVM inside an SGX enclave would currently consist in significant re-engineering efforts.

Few recent contributions tackle privacy-preserving data processing, particularly in a MapReduce scenario.
This is the case of Airavat~\cite{Roy:2010:ASP:1855711.1855731} or GUPT~\cite{Mohan:2012:GPP:2213836.2213876}.
These systems leverage differential-privacy techniques~\cite{dwork2006calibrating} and can face a different threat model than the one supported by SGX and hence by \SYS.
In particular, one deploying such systems on a public infrastructure, one needs to trust the provider. 
Our system greatly reduces the trust boundaries, and only require to trust Intel.  

To best of our knowledge, \SYS{} is the first lightweight and low-memory footprint stream processing framework that can fully execute within SGX enclaves.
