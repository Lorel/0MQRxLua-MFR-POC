\section{Architecture}
\label{sec:architecture}
\begin{figure*}[!t]
  \centering
  \includegraphics[scale=0.5]{images/architecture_pipeline}
  \caption{Example of \SYS pipeline architecture.}
  \label{fig:architecture_pipeline}
\end{figure*}

% \begin{itemize}
%   \item Communication by ØMQ (version 4.1.4)
%   \item Docker container for clustered deployment
%   \item Workers connected by routers: define worker's role and router's operating
%   \item Fire-and-forget messaging: a messsaging pattern in which we do not expect a direct response to the message, as opposed to request-response protocols
%   \item Based on Lua
% \end{itemize}

The architecture of \SYS{} comprises a combination two different types of components: \textsf{worker} and \textsf{router}.
A worker component continuously listens for incoming data by means of non-blocking I/O.
As soon as data flows in, some application-dependent business logic is applied.
A typical use-case is the deployment of a classic filter/map/reduce pattern from the functional programming paradigm~\cite{bird_introduction_1988}.
In that case, worker nodes execute only one function, namely \texttt{map}, \texttt{filter} or \texttt{reduce}.
A router acts as a message broker between workers in the pipeline and transfers data between workers according to some dispatching policy.
Figure~\ref{fig:architecture_pipeline} depicts a possible implementation of this dataflow pattern using the \SYS middleware.
%Datas in \textsc{SecureStreams} are streamed accross the process pipeline as shown in figure \ref{fig:architecture_pipeline}.

\SYS is designed to allow processing of sensible data inside SGX enclaves.
As explained in the previous section, the \emph{Enclave Page Cache} (EPC) is currently limited at 128\,MB.\footnote{Note that future releases of SGX might relax this limitation~\cite{mckeen2016intel}.}
To overcome these limitations, we settled on a lightweight yet efficient embeddable runtime based on the Lua Virtual Machine~\cite{ierusalimschy_luaextensible_1996} and the corresponding multi-paradigm scripting language~\cite{lualang}.
Note that the Lua runtime requires only few kilobytes of memory, and thus representing an ideal candidate to execute in the limited space allowed by the EPC.
Moreover, the application-specific functions can be quickly prototyped in Lua, and even complex algorithms can be implemented with an almost 1:1 mapping from pseudo-code~\cite{leonini2009splay}.
%If a process exceeds the available memory, an encrypted pagination mechanism leads to performance leaks.
%Thus \textsc{SecureStreams} has been designed to use a Lua runtime.
%Lua is a lightweight multi-paradigm programming language designed primarily for embedded systems and clients\cite{ierusalimschy_luaextensible_1996}.
%Its runtime requires only few KB of memory, and thus fits easily in EPC.

Each component is wrapped inside a lightweight Linux container (in our case, Docker).
This choice facilitate the embedding of all required dependencies, as well as the correctness of their configuration, within an isolated and reproducible execution environment.
By doing so, a \SYS processing pipeline can be easily deployed without changing the source code on different infrastructures, such as the Amazon EC2 Container Service~\cite{awsec2container}, either on a single machine or a cluster, using a Docker network and the Docker Swarm\cite{docker:swarm_2016} scheduler.

The communication between workers and routers leverages \zmq, a high-performance asynchronous messaging library~\cite{zero_mq}.
Each router component hosts inbound and outbound queues. %\vs{how much?}\ah{We don't know, so we should not talk about queue size. BTW, I don't understand your sentence. Do you missed a verb?}.
In particular, we use the \zmq's pipeline pattern with the \textsc{Push}-\textsc{Pull} protocol~\cite{zero_mq:pipeline}.
The inbound queue is a \textsc{Pull} socket.
The messages are streamed from a set of anonymous\footnote{\emph{Anonymous} is said about a peer without any identity: the server socket ignore which worker sent the message.} \textsc{Push} peers (\emph{e.g.}, the upstream workers in the pipeline).
The inbound queue uses a fair-queuing scheduling to deliver the message to the upper layer.
% dispatches the messages.\ah{that's not true, the fair-queuing algorithm operates at the level of the reception of the queue, and it does not dispatch anything, the message is passed to the outgoing socket/queue, that's it}
Conversely, the outbound queue is a \textsc{Push} socket, sending messages using a round-robin algorithm to a set of anonymous \textsc{Pull} peers, the downstream workers.
% \vs{if there is time, it could be useful to have a drawing that zooms into this aspect of the architecture, not the full pipeline}
This design allows to dynamically scale up and down each stage of the pipeline to adapt to the needs of the application or workload. %It is scalable in that nodes can join at any time.
Finally, \zmq guarantees that the messages are delivered across each stage via reliable TCP channels.
%The pattern is mostly reliable insofar as it will not discard messages unless a node disconnects unexpectedly.
%This fire-and-forget messaging is a messsaging pattern in which we do not expect a direct response to the message, as opposed to request-response protocols\cite{voelter_patterns_2003}.
% The absence of response to a message provides some relevant performances.

\vs{Add some paragraphs to  detail how  the interaction with the SGX enclaves work in the context of \SYS}

\vs{It should be useful to describe how the dataflow pipeline is mapped to the underlying cluster.}
