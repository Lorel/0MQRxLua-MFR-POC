\documentclass{sig-alternate}

\usepackage[utf8]{inputenc}
\usepackage{graphicx}
\usepackage{subfig}
\usepackage[usenames,dvipsnames,svgnames,table]{xcolor}
\usepackage{amsmath,amssymb}
\usepackage{ifthen}
\usepackage{xspace}
\usepackage{cite}

\usepackage{algorithm}
%\usepackage[noend]{algpseudocode}
\usepackage{algpseudocode}
\usepackage{color}
\usepackage{listings}
\usepackage{pifont}

\usepackage{url}
\usepackage[colorlinks=false,linkcolor=blue,urlcolor=blue,citecolor=blue,bookmarks=false,draft]{hyperref}
\renewcommand*{\UrlFont}{\ttfamily\rm}

\usepackage{listings}

\lstset{
  float,
  basicstyle       = \scriptsize\ttfamily,
  columns          = fixed,
  numbers          = left,
  numberstyle      = \tiny\ttfamily,
  numbersep        = 5pt,
  xleftmargin      = 5pt,
  xrightmargin     = 10pt,
  frame            = tb,
  captionpos       = b,
  upquote          = true,
  showstringspaces = false,
}


\newcommand{\comment}[1]{}

% \usepackage{array}
% \usepackage{multirow}
% \usepackage{paralist}
% \usepackage{ragged2e}

% \usepackage[subtle]{savetrees}

% \usepackage[small,compact,noindentafter]{titlesec}
% \titlespacing\section{0pt}{6pt plus 2pt minus 2pt}{2pt plus 2pt minus 2pt}
% \titlespacing\subsection{0pt}{4pt plus 2pt minus 2pt}{2pt plus 1pt minus 1pt}
% \titlespacing{\paragraph}{0pt}{2pt plus 0pt minus 1pt}{1.0ex}

% \usepackage{setspace}
% \usepackage[margin=8pt,font={small,bf,stretch=0.85}]{caption}

% \usepackage{enumitem}
% \setitemize{noitemsep,topsep=3pt,parsep=3pt,partopsep=3pt}

% \usepackage[inline]{enumitem}
% \setlist{noitemsep,topsep=0pt,parsep=0pt,partopsep=0pt}

% \def\arraystretch{0.9}

\newcommand{\SYS}{\textsc{SecureStreams}\xspace}
\newcommand{\zmq}{\textsc{ZeroMQ}\xspace}
\newcommand{\rxl}{\textsc{RxLua}\xspace}

\setlength{\textfloatsep}{10pt}

\newboolean{showcomments}
\setboolean{showcomments}{true}
\ifthenelse{\boolean{showcomments}}
{ \newcommand{\mynote}[3]{
   \fbox{\bfseries\sffamily\scriptsize#1}
   {\small$\blacktriangleright$\textsf{\emph{\color{#3}{#2}}}$\blacktriangleleft$}}}
{ \newcommand{\mynote}[3]{}}
\newcommand{\pf}[1]{\mynote{Pascal}{#1}{pink}}
\newcommand{\vs}[1]{\mynote{Valerio}{#1}{blue}}
\newcommand{\ah}[1]{\mynote{Aurelien}{#1}{red}}
\newcommand{\rp}[1]{\mynote{Rafael}{#1}{orange}}
\newcommand{\rr}[1]{\mynote{Romain}{#1}{green}}


\definecolor{darkgreen}{rgb}{0.3,0.5,0.3}
\definecolor{darkblue}{rgb}{0.3,0.3,0.5}
\definecolor{darkred}{rgb}{0.5,0.3,0.3}
\lstdefinelanguage{LUA}{
  sensitive=true,
  keywordstyle=[1]{\color{darkblue}\bfseries},
  keywordstyle=[2]{\color{darkgreen}\bfseries},
  morekeywords=[1]{and,break,do,else,elseif,end,for,function,if,in,local,
    nil,not,or,repeat,return,then,until,while,require,alias},% Official LUA keywords
  morekeywords=[2]{},% Your private keywords
  otherkeywords={.,=,~,*,>,:},
  morestring=[b]",
  stringstyle={\color{darkred}\itshape},
  breaklines=true,
  breakatwhitespace=true,
  linewidth=\columnwidth,
  comment=[l]{--},
  escapeinside={(*@}{@*)}
}

\lstdefinelanguage{YAML}{
  sensitive=true,
  keywordstyle=[1]{\color{darkblue}\bfseries},
  keywordstyle=[2]{\color{darkgreen}\bfseries},
  morekeywords=[1]{image, entrypoint, environment, devices, hostname},% Official LUA keywords
  morekeywords=[2]{},% Your private keywords
  otherkeywords={.,=,~,*,>,:},
  morestring=[b]",
  breaklines=true,
  breakatwhitespace=true,
  linewidth=\columnwidth,
  comment=[l]{--},
  escapeinside={(*@}{@*)}
}

\begin{document}

% \title{\textsc{0MQRxLua}: Middleware for Reactive and Secure Data Stream Processing}
% \title{SecureStreams: Scalable Middleware for Reactive and Secure Data Stream Processing}
\title{SecureStreams: Reactive Middleware for\\Secure Data Stream Processing}
\date{}

\author{
%\IEEEauthorblockN{Aur\'elien Havet\IEEEauthorrefmark{1}, Valerio Schiavoni\IEEEauthorrefmark{1}}
%\IEEEauthorblockA{\IEEEauthorrefmark{1}University of Neuchâtel, Switzerland. Email: first.last@unine.ch}
%%\IEEEauthorblockA{\IEEEauthorrefmark{2}Univ. Lille / Inria / IUF, France. Email: romain.rouvoy@univ-lille.fr}
}
\sloppy
\maketitle

%\thispagestyle{empty}

\begin{abstract}
The growing adoption of distributed data processing frameworks in a wide diversity of application domains challenges the integration of properties like security, in particular when considering deployments in the context of large-scale clusters or multi-tenant Cloud infrastructures.

This paper therefore introduces \SYS, a reactive middleware framework to deploy and process secure streams at scale.
Its design combines the reactive dataflow programming paradigm with Intel's \emph{Software Guard Extensions} (SGX) to guarantee the privacy and the integrity of the data being processed.
The experimental results of \SYS are promising: while offering a fluent scripting language based on \textsc{Lua}, our middleware delivers high processing throughput, thus enabling developers to implement complex processing pipelines in a few lines of code.
\end{abstract}

%\begin{IEEEkeywords}
%streaming; reactive programming; Lua; SGX
%\end{IEEEkeywords}

\section{Introduction}
\label{sec:introduction}
\vs{recheck the storyline and reframe where needed. Maybe stress more the security aspect and how TEE can help}

Data streams are more and more predominant in today's era of big data.
In the world of all-connected and the \emph{Internet-of-Things} (IoT), on market places, or elsewhere, data is continuously produced and consumed.
%The latest estimates predict a total of \vs{fill} TB/s to be analyzed efficiently and securely within the next 10 years.
However, such data streams requires more and more to be processed with reliability, scalability and security in mind.
This paper introduces \SYS{}, our initial work on a middleware framework for developing secure stream processing in the Cloud.
\SYS{} supports the implementation, deployment and the execution of stream processing tasks in distributed settings, from clusters to large-scale Cloud infrastructures.
\SYS{} adopts of message-oriented~\cite{mom}, responsive, resilient to faults middleware design.
Finally, it scales horizontally and vertically seamlessly \rp{SGX is limited in memory usage, which weakens the claim of vertical scalibity}.
%Its design This middleware is designed according to The Reactive Manifesto: it is responsive, resilient, elastic and message driven\cite{reactivemanifesto}.
Briefly, its design is inspired by the dataflow programming paradigm~\cite{uustalu_essence_2005}: the developer combines together several independent processing components (\emph{e.g.}, mappers, reducers, sinks, shufflers, joiners) to compose several processing pipes.% an easy way to implement a processing pipe.
The framework integrates an abstraction of the required deployment infrastructure.
It smoothly integrate with industrial-grade lightweight virtualization technologies (\emph{e.g.}, Docker~\cite{docker}).
%by deployment automation based on the Docker ecosystem
It aims to be secure: communication channels use the SSL protocol~\cite{freier2011secure} for data communication.
Finally, our design intends to exploit the secure processing capabilities offered by trusted hardware \emph{enclaves}, nowadays widely available into mass-market thanks to the introduction of Intel's \emph{Software Guard eXtensions} (SGX)~\cite{costan_intel} in the SkyLake processors~\cite{skylake}.
% and giving the ability to process datas in trusted enclaves by the integration of the Intel's Software Guard Extension (SGX) hardware.

Few mainstream solutions exist today for distributed stream processing.
Reactive Kafka~\cite{reactivekafka} allows stream processing atop of Apache Kafka~\cite{apachekafka}.
More recently, few open-source solutions (\emph{e.g.}, Apache Spark Streaming~\cite{apachesparkstreaming}, Apache Storm~\cite{apachestorm}, Infinispan~\cite{infinispan}) introduced APIs to allow developers to quickly setup and deploy stream processing infrastructures.
These systems rely on the \emph{Java Virtual Machine} (JVM)~\cite{lindholm2014java}.
However, SGX currently imposes a hard memory limit of 128\,MB to the enclaved code and data, at the cost of expensive encrypted memory paging mechanisms and serious performance leaks~\cite{pires_scbr:2016,brenner_securekeeper:_2016} when this limit is crossed.
\SYS{} proposes a lightweight and low-memory footprint framework that can fully execute within SGX enclaves.
We detail our implementation choices in Section~\ref{sec:implementation}.

%But one of our goals is to find out a software technology that can run in a SGX enclave, and the memory of the latter is limited to 96MB for the Intel Skylake processor (the first and only processor including SGX currently)\cite{costan_intel}.
%If the application running inside an enclave needs more memory, an expensive encrypted memory paging mechanism then is used, causing serious performance leaks~\cite{brenner_securekeeper:_2016}.
To recap, our \textbf{contributions} are the followings: (i) an architecture of \SYS{}, (ii) details of our implementation choices, and (iii) results of our preliminary evaluation based on a real-world dataset.

This paper is organized as follows.
The architecture of \SYS{} is described in Section~\ref{sec:architecture}.
Our implementation choices and an example of \SYS{} program are presented in Section~\ref{sec:implementation}.
Section~\ref{sec:eval} reports on our preliminary evaluation with respect to throughput and scalability results.
Before concluding, we wrap up by briefly describing our future work in Section~\ref{sec:future}.


%\section{Motivating scenarios}
\label{sec:motivating}
\vs{here we can explain and provide details of the use-cases. One shoudl come from SecureCloud, maybe another one from well-known stream-processing benchmarks}

\section{Background on Intel SGX}\label{sec:background}
The design of \SYS{} revolves around the availability of SGX features in the host machines.
It consists in a \emph{trusted execution environment} (TEE) recently introduced into Intel SkyLake, similar in spirit to ARM \textsc{TrustZone}~\cite{arm2009security} but much more powerful.
Applications create secure \emph{enclaves} to protect the integrity and the confidentiality of the data and the interpreted code being executed. 
%\rp{There is no confidentiality on the binary code than runs in the enclave. On the other hand, we encrypt interpreted code, in which case the claim holds true.}
% 
The SGX mechanism allows applications to access confidential data from inside the enclave. 
The architecture guarantees that an attacker with physical access to a machine will not be able to tamper with the application data without being noticed. 
% The CPU package represents the security boundary. 
Moreover, data belonging to an enclave is automatically encrypted and authenticated when stored in main memory. 
% A memory dump on a victim’s machine will produce encrypted data.
% A \emph{remote attestation protocol} allows one to verify that an enclave runs on a genuine Intel processor with SGX.
% An application using enclaves must ship a signed (not encrypted) shared library (a shared object file in Linux) that can possibly be inspected by malicious attackers.
% 
In SGX, the \emph{enclave page cache} (EPC) is a $128\,MB$ area of memory predefined at boot to store enclaved code and data. 
At most around $90\,MB$ can be used by application’s memory pages, while the remaining area is used to maintain SGX metadata. 
Any access to an enclave page that does not reside in the EPC triggers a page fault.
The SGX driver interacts with the CPU to choose which pages to evict. 
The traffic between the CPU and the system memory is kept confidential by the \emph{memory encryption engine} (MEE)~\cite{gueron2016memory}, also in charge of tamper resistance and replay protection. 
If a cache miss hits a protected region, the MEE encrypts or decrypts data before sending to or fetching from the memory and performs integrity checks. 
% Data can also be persisted on stable storage protected by a seal key. 
% This allows the storage of certificates, waiving the need of a new remote attestation every time an enclave application restarts.

% \begin{figure}[!t]
%   \centering
%   \includegraphics[width=\linewidth]{images/sgx}
%   \caption{SGX core operating principles.}
%   \label{fig:sgx}
% \end{figure}

% The execution flow of a program using SGX enclaves is like the following.
% First, an enclave is created (see Figure~\ref{fig:sgx}-\ding{202}).
% As soon as a program needs to execute a trusted function (\ding{203}), it executes SGX's primitive \texttt{ecall} (\ding{204}).
% The call goes through the SGX call gate to bring the execution flow inside the enclave (\ding{205}).
% Once the trusted function is executed by one of the enclave's threads (\ding{206}), its result is encrypted and sent back (\ding{207}) before giving back the control to the main processing thread (\ding{208}).


\section{Architecture}
\label{sec:architecture}
\begin{figure*}[!t]
  \centering
  \includegraphics[scale=0.5]{images/architecture_pipeline}
  \caption{Example of \SYS pipeline architecture.}
  \label{fig:architecture_pipeline}
\end{figure*}


The architecture of \SYS{} comprises a combination of two different types of components: \textsf{worker} and \textsf{router}.
A worker component continuously listens for incoming data by means of non-blocking I/O.
As soon as data flows in, an application-dependent business logic is applied.
A typical use-case is the deployment of a classic filter/map/reduce pattern from the functional programming paradigm~\cite{bird_introduction_1988}.
In such case, worker nodes execute only one function, namely \texttt{map}, \texttt{filter} or \texttt{reduce}.
A router acts as a message broker between workers in the pipeline and transfers data between them according to a given dispatching policy.
Figure~\ref{fig:architecture_pipeline} depicts a possible implementation of this dataflow pattern using the \SYS middleware.

\SYS is designed to allow processing of sensible data inside SGX enclaves.
As explained in the previous section, the \emph{Enclave Page Cache} (EPC) is currently limited at 128\,MB.\footnote{Future releases of SGX might relax this limitation~\cite{mckeen2016intel}.}
To overcome this limitation, we settled on a lightweight yet efficient embeddable runtime, based on the Lua Virtual Machine~\cite{ierusalimschy_luaextensible_1996} and the corresponding multi-paradigm scripting language~\cite{lualang}.
The Lua runtime requires only few kilobytes of memory, it is designed to be embeddable, and as such it represents an ideal candidate to execute in the limited space allowed by the EPC.
Moreover, the application-specific functions can be quickly prototyped in Lua, and even complex algorithms can be implemented with an almost 1:1 mapping from pseudo-code~\cite{leonini2009splay}.
We provide further implementation details of the embedding of the LuaVM inside an SGX enclave in Section~\ref{sec:implementation}.

%\rp{I would argue that the choice of Lua is not actually to overcome the memory limitation. I guess the majority of script interpreters would fit easily inside the 90MB. Besides, *ideal* is a quite strong claim. Of course we do need a script language since we cannot link code dinamically, but I would rather go with the programability arguments, the availability of RxLua, besides of course being easily embeddable and tiny.}

%If a process exceeds the available memory, an encrypted pagination mechanism leads to performance leaks.
%Thus \textsc{SecureStreams} has been designed to use a Lua runtime.
%Lua is a lightweight multi-paradigm programming language designed primarily for embedded systems and clients\cite{ierusalimschy_luaextensible_1996}.
%Its runtime requires only few KB of memory, and thus fits easily in EPC.

Each component is wrapped inside a lightweight Linux container (in our case, the defacto industrial standard Docker~\cite{docker}).
Each container embeds all the required dependencies, while guaranteeing the correctness of their configuration, within an isolated and reproducible execution environment.
By doing so, a \SYS processing pipeline can be easily deployed without changing the source code on different public or private infrastructures.
For instance, this will allow to deploy \SYS on Amazon EC2 Container Service~\cite{awsec2container}, where soon SkyLake-enabled instances will be made available~\cite{amazonskylake}, or similarly to Google Compute Engine~\cite{gceskylake}.
The deployment of the containers can be transparently executed on a single machine or a cluster, using a Docker network and the Docker Swarm\cite{docker:swarm_2016} scheduler.
%\rp{Does Amazon offer machines with Skylake processors with EPC enabled?}
%\ah{It's planned: https://aws.amazon.com/fr/about-aws/whats-new/2016/11/coming-soon-amazon-ec2-c5-instances-the-next-generation-of-compute-optimized-instances/ but we don't know if SGX will be enabled. Google intends also: http://fortune.com/2017/02/24/google-intel-cloud-chip/. Do we have to cite there references?}

The communication between workers and routers leverages \zmq, a high-performance asynchronous messaging library~\cite{zero_mq}.
Each router component hosts inbound and outbound queues.
In particular, the routers use the \zmq's Pipeline pattern~\cite{zero_mq:pipeline} with the \textsc{Push}-\textsc{Pull} socket types. 
%\rp{Is 'protocol' the best definition? I would rather call it a 'pattern'}\ah{It is, this is one of the different protocols of communication implemented in ZMQ, like explained in the target of the given pointer}\rp{The pointer does not mention protocol to refer to that. push/pull refers to the role of socket endpoints. It is not an agreement on format and content of exchanged messages, as a 'protocol' would be best understood by a common reader.}



The inbound queue is a \textsc{Pull} socket.
The messages are streamed from a set of anonymous\footnote{\emph{Anonymous} is said about a peer without any identity: the server socket ignore which worker sent the message.} \textsc{Push} peers (\emph{e.g.}, the upstream workers in the pipeline).
The inbound queue uses a fair-queuing scheduling to deliver the message to the upper layer.
Conversely, the outbound queue is a \textsc{Push} socket, sending messages using a round-robin algorithm to a set of anonymous \textsc{Pull} peers, \emph{e.g.} the downstream workers.
\begin{lstlisting}[language=YAML,caption={\SYS pipeline examples. Some attributes (\texttt{volume}, \texttt{networks}, \texttt{env\_file}) omitted.},label=pipeline-desc][!t]
sgx_mapper:
  image: "${IMAGE_SGX}"
  entrypoint: ./start.sh sgx-mapper.lua
  environment:
    - TO=tcp://router_mapper_filter:5557
    - FROM=tcp://router_data_mapper:5556
    - "constraint:type==sgx"
  devices:
    - "/dev/isgx"

router_data_mapper:
  image: "${IMAGE}"
  hostname: router_data_mapper
  entrypoint: lua router.lua
  environment:
    - TO=tcp://*:5556
    - FROM=tcp://*:5555
    - "constraint:type==sgx"

data_stream:
  image: "${IMAGE}"
  entrypoint: lua data-stream.lua
  environment:
    - TO=tcp://router_data_mapper:5555
    - "constraint:type==sgx"
    - DATA_FILE=the_stream.csv
\end{lstlisting}


% \vs{if there is time, it could be useful to have a drawing that zooms into this aspect of the architecture, not the full pipeline}
This design allows to dynamically scale up and down each stage of the pipeline in order to adapt it to application's needs or the workload.
Finally, \zmq guarantees that the messages are delivered across each stage via reliable TCP channels.
%The pattern is mostly reliable insofar as it will not discard messages unless a node disconnects unexpectedly.
%This fire-and-forget messaging is a messsaging pattern in which we do not expect a direct response to the message, as opposed to request-response protocols\cite{voelter_patterns_2003}.
% The absence of response to a message provides some relevant performances.


We define the processing pipeline components and their chaining by means of the Docker Compose's custom description language~\cite{docker:compose}.
Listing~\ref{pipeline-desc} shows a snippet of the description used to deploy the architecture in Figure~\ref{fig:architecture_pipeline}.
Once the processing pipeline is defined, the containers must be deployed on the computing infrastructure.
We exploit the \texttt{constraint} placement mechanism to enforce the Docker Swarm's scheduler to deploy workers requiring SGX capabilities into appropriate hosts.
In the example, an \texttt{sgx\_mapper} nodes is deployed on an SGX host by specyfing \texttt{"constraint:type==sgx"} in the Compose description.



%\vs{Add some paragraphs to  detail how  the interaction with the SGX enclaves work in the context of \SYS}
%\vs{It should be useful to describe how the dataflow pipeline is mapped to the underlying cluster.}


\section{Architecture}
\label{sec:architecture}

\begin{itemize}
  \item Extension of the library RxLua
  \item ØMQ Lua bindings
  \item ØMQ queues with push/pull protocol - Fire-and-forget messaging: a messsaging pattern in which we do not expect a direct response to the message, as opposed to request-response protocols
\end{itemize}


% \lstset{language=Lua,caption={Descriptive Caption Text},label=DescriptiveLabel}

\begin{lstlisting}[frame=single]
Rx.Observable.fromTable(people)
  :map(
    function(person)
      return person.age
    end
  )
  :filter(
    function(age)
      return age > 18
    end
  )
  :reduce(
    function(accumulator, age)
      accumulator[count] = (accumulator.count or 0) + 1
      accumulator[sum] = (accumulator.sum or 0) + age
      return accumulator
    end, {}
  )
  :subscribe(
    function(datas)
      print('Adult people avverage:', datas.sum / datas.count)
    end,
    function(error)
      print(error)
    end,
    function()
      print('Process complete!')
    end
  )
\end{lstlisting}


\section{Preliminary Evaluation}
\label{sec:eval}

% \begin{itemize}
%   \item Experiment on datas from the Bureau of Transportation Statistic (\url{http://www.transtats.bts.gov/Fields.asp?Table_ID=236}) to evaluate flight delays by airport
%   \item Infrastructure based on ONE cluster, VM 16.04 LTS (GNU/Linux 4.4.0-53-generic x86\_64), Docker 1.13.0-rc3, Docker Swarm 1.2.5 (discovery: Consul v0.5.2)
%   \item Hardware?
% \end{itemize}
This section presents our preliminary evaluation of \SYS.
First we present our evaluation settings, then the dataset and finally some preliminary benchmark, namely throughput and scalability.

\textbf{Evaluation settings.} We deploy a cluster of virtual machines (VM) based on Ubuntu 16.04 LTS and running a daemon Docker (v.1.13.0-rc3).
Each VM is set with 2 CPU cores and 2GB RAM, and interconnected using a switched 1~Gbps network.
Nodes join a Docker Swarm~\cite{}\vs{FIX} (v1.2.5), and Consul 0.5.2~\cite{}\vs{FIX} as discovery service.
Each VM only execute one single Docker instance, to prevent cross-container interferences\vs{we should add a ref about this}. 
%A single-one container is launch on each node.
Containers leverage the Docker overlay network to comunicate to each other.

\textbf{Dataset.} In our experiments we process a real dataset released by the American Bureau of Transportation Statistic\cite{rita:bts}.
%including each flight's departure and arrivals\cite{statistical_computing:data}.
The dataset reports the flight departures and arrivals of 20 air carriers\cite{statistical_computing:data}.%\ah{pointers on datas are not relevant?}
We implement a simple application on top of \SYS to determine average delays and the total of delayed flights for each air carrier.
%These datas report flights departures and arrivals\cite{rita:bts} and are available on the Statistical Computing\cite{statistical_computing:data}.
We design and implement a simple processing pipeline, that (i) parses the input datasets (in a comma-separated-value format) to data structure (map), (ii) filters by relevancy (i.e. if the data concerns a delayed flight), and (iii) finally reduces to compute to obtain the required informations.\footnote{This experiment is inspired by Kevin Webber's blog entry \emph{Diving into Akka Streams}: \url{https://blog.redelastic.com/diving-into-akka-streams-2770b3aeabb0}.}
We use the 4 last years of the available dataset (from 2005 to 2008), for a total of 28 millions of entries to process and a total size of 2.73 GB.

\textbf{Benchmark: throughput}
This benchmark shows the upload throughtput observed across all cluster while streaming the dataset as fast as possible from the source nodes into the processing pipeline.
We gather banwdwidth measurements by exploting Docker's own monitoring and statistical modules.
%Throughput accross containers wrapping each node of the processing pipeline are measured from Docker stats.
The statistics are gathered at runtime while the experiment is executing.
%During the experiment, we retrieve all the data stats for each container.
%In particular, \texttt{txbytes} stats are extracted to measure containers output throughput.
We present our results in Figure~\ref{fig:throughput}.
In this scenario, only one input source injects the input data into the processing pipeline.
We use a representation based on stacked percentiles. 
The white bar at the bottom represents the minimum value, the pale grey on top the maximal value. 
Intermediate shades of grey represent the 25th, 50th–the median–, and 75th percentiles. 
For instance, the median throughput at 50 seconds into the experiment is at 1500 kB/s, meaning that 50\% of the nodes output data at 1500 kB/s or less.
%These datas are computed to be plotted together by percentile, as shown on figure \ref{fig:throughput}.\ah{maybe it could be relevant to put three plots, corresponding to experiments 4-datas-1-worker, 4-datas-2-workers and 4-datas-4-workers}
With the current implementation, we obseve a peak of 10MB/s upload throughput the processing stages.\vs{for the future, it'd be interesting to know which stage is the fastest one}
%\ah{I gonna measure throuput between two containers run on a swarm cluster, using iperf, for comparison purpose}


\begin{figure}[t!]
  \centering
  \includegraphics[scale=0.7]{images/tput_upload}
  \caption{Upload Throughput. The middleware completes the processing of the dataset in 305\vs{CHECK} seconds, with a peak of 10 MB/s and an overall throughput of 8.9 MB/s}
  \label{fig:throughput}
\end{figure}

\textbf{Benchmark: scalability}
We also include preliminary scalability results of the \SYS framework.
In particular, we scale up each stage of the processing pipeline, with 1, 2 or 4 workers.
Results are represented on figure \ref{fig:scalability}. 
We present average and standard deviation of the overall completion time of the job, by executing the experiment 20 times for each configuration.
%Scalability of \SYS is evaluated by processing these datas 20 times on 3 different pipeline topology: using 1, 2 or 4 workers for each step of the pipeline.
We observe that by doubling the number of workers from the initial configuration half in two the overall processing time, from 20 minutes to less than 10 minutes.
Convesely, we do not observe these improvements when using 4 workers.
%Results are represented on figure \ref{fig:scalability}, and show clearly better performances between the experiment using only one worker by task, and the one using 2 workers.
%In an other hand, using 4 workers instead of 2 does not show any performance improvement.

\begin{figure}[t!]
  \centering
  \includegraphics[scale=0.5]{images/avg_stdev_4_streams}
  \caption{Scalability.}
  \label{fig:scalability}
\end{figure}

The latter observation can be explained by the fact that each job in our experiment is very simple and executed too quickly by the workers, compared to the throughput capacity of the communication between two steps of the process pipeline.
This may be due to the limitation of the network bandwith capacity, when the cost of data communication accross nodes is higher than the one of data computation, as described by the Gunther's Universal Law of Computational Scalability\cite{gunther1993simple}, where the relative capacity of a computational platform is inversely proportional to the sum of the levels of contention (e.g., queueing for shared resources) and coherency delay (i.e., latency for data to become consistent) in the system.
In an other hand, it may also be due to the limitation of the router bandwith, the improvement of which is a part of our future work.


\section{Related Work}\label{sec:rw}

\vs{to do from scratch, should cover: stream processing papers, event-based middlewares, middleware that exploits hardware features}

A few dedicated solutions exist today for distributed stream processing using reactive programming.
For instance, \textsc{Reactive Kafka}~\cite{reactivekafka} allows stream processing atop of Apache \textsc{Kafka}~\cite{apachekafka}.
These solutions do not, however, support secure execution in a trusted execution environment.

More recently, some open-source middleware frameworks (\emph{e.g.}, Apache \textsc{Spark}~\cite{apachesparkstreaming}, Apache \textsc{Storm}~\cite{apachestorm}, \textsc{Infinispan}~\cite{infinispan}) introduced APIs to allow developers to quickly set up and deploy stream processing infrastructures.
These systems rely on the \emph{Java} virtual machine (JVM)~\cite{lindholm2014java}.
However, SGX currently imposes a hard memory limit of 128\,MB to the enclaved code and data, at the cost of expensive encrypted memory paging mechanisms and serious performance overheads~\cite{pires_scbr:2016,brenner_securekeeper:_2016} when this limit is crossed.
Moreover, executing a fully-functional JVM inside an SGX enclave would currently consist in significant re-engineering efforts.

To best of our knowledge, \SYS{} is the first lightweight and low-memory footprint stream processing framework that can fully execute within SGX enclaves.


\section{Future work}
\label{sec:future}
% \begin{itemize}
%   \item Fix bottleneck in routers
%   \item Ability to embed Lua libraries and C or C++ components
%   \item Implements TLS/SSL communication with Curve
%   \item SGX integration
%   \item Architecture automation using Docker API
% \end{itemize}
We plan to extend \SYS along the following directions, with the common goal of making the middleware framework more secure.
First, the communication between the processing components will rely on secure channels.
%At the moment \SYS is not yet secured.
%To do this, we have to secure both the communication and the processing.
To this end, we plan to integrate \textsc{CurveZMQ}~\cite{zmq:curvezmq}, an open-source authentication and encryption protocol for \zmq.
\textsc{CurveZMQ} is based on a fast, secure elliptic-curve cryptographic primitive provided by the libraries \textsc{CurveCP} and \textsc{NaCl}.
Moreover, \SYS will generate the authentication certificates in use by the containers on-the-fly at the moment of their deployment.

Secondly, the \SYS will exploit the SGX enclaves to both process and route data.
The computation will be executed inside a trusted enclave, ensuring the encryption of the data and the integrity of the executed code.

We intend to evaluate the impact of these modifications on the performance of the system, in particular with respect to the overall throughput, memory and CPU usage.

The current implementation of \SYS does not provide full automation of container deployments yet.
In particular, we will supersede the current approach based on Docker Compose and instead will rely on the more portable Docker APIs.

Finally, \SYS will allow to embed Lua scripts or native (\emph{e.g.}, C/C++) components inside a node of the processing pipeline.
This feature will improve the usability of the framework.
%Thus, those libraries or components will be callable directly from the code given in an element of the pipeline.


\section{Conclusion}
\label{sec:conclusion}
Secure stream processing is becoming a major concern in the era of IoT.
This paper introduces our design and evaluation of \SYS, an easy-to-use and efficient framework to implement, deploy and evaluate stream processing pipelines for continuous data streams.
The framework is designed to exploit the Trusted Execution Environments nowadays available in Intel's processors such as the latest SkyLake.
We implemented the prototype of \SYS in Lua and based its APIs around the reactive programming approach.
Our initial evaluation results based on real-world traces are encouraging.
We plan to further extend and throughly evaluate \SYS in our future work.


%double-blind, no acks before CR
%\section*{Acknowledgments}
%Aurelien Havet's PhD research is co-supervised by Pascal Felber from the University of Neuchâtel and Romain Rouvoy from the University of Lille.
%The research leading to these results has received funding from the European Commission, Information and Communication Technologies, H2020-ICT-2015 under grant agreement number 690111 (SecureCloud project).
%
{
%\footnotesize
\bibliographystyle{acm}
\bibliography{biblio}
% \balancing
}

\end{document}
