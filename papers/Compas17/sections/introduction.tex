\section{Introduction}\label{sec:introduction}
The data deluge imposed by a world of ever-connected devices, whose most emblematic example is the \emph{Internet of things} (IoT), has fostered the emergence of novel data analytics and processing technologies to cope with the ever increasing \emph{volume}, \emph{velocity}, and \emph{variety} of information that characterize the big data era.
In particular, to cope with the continuous flow of information gathered by millions of IoT devices, data streams have emerged as a suitable paradigm to process flows of data at scale.
However, as some of these data streams may convey sensitive information, stream processing requires to support end-to-end security guarantees in order to prevent third parties to access restricted data.

This paper therefore introduces \SYS{}, our initial work on a middleware framework for developing and deploying secure stream processing on untrusted distributed environments.
\SYS{} supports the implementation, deployment, and execution of stream processing tasks in distributed settings, from large-scale clusters to multi-tenant Cloud infrastructures.
More specifically, \SYS{} adopts a message-oriented~\cite{mom} middleware, which integrates with the SSL protocol~\cite{freier2011secure} for data communication and the current version of Intel's \emph{software guard extensions} (SGX)~\cite{costan_intel} to deliver end-to-end security guarantees along data stream processing stages.
% \SYS{} can scale vertically and horizontally by adding or removing processing nodes at any stage of the pipeline, for example to dynamically adjust according to the current workload.
The design of the \SYS{} system is inspired by the dataflow programming paradigm~\cite{uustalu_essence_2005}: the developer combines together several independent processing components (\emph{e.g.}, mappers, reducers, sinks, shufflers, joiners) to compose specific processing pipes.
Regarding packaging and deployment, \SYS{} smoothly integrates with industrial-grade lightweight virtualization technologies like Docker~\cite{docker}.

% In this paper, we propose the following contributions: (i)~we describe the design of \SYS, (ii)~we provide details of our reference implementation, in particular on how to smoothly integrate our runtime inside an SGX enclave, and (iii) we perform an extensive evaluation with micro-benchmarks, as well as with a real-world dataset.

The remainder of the paper is organized as follows.
To better understand the design of \SYS, Section~\ref{sec:background} delivers a brief introduction to today's SGX operating mechanisms.
The architecture of \SYS{} is then introduced in Section~\ref{sec:architecture}.
Our implementation choices and an example of \SYS{} program are reported in Section~\ref{sec:implementation}.
Section~\ref{sec:eval} discusses our extensive evaluation, based on macro-benchmarks with real-world datasets.
Finally, Section~\ref{sec:rw} briefly discusses the state-of-the-art before concluding in Section~\ref{sec:conclusion}.
