\section{Introduction}
\label{sec:introduction}

% \begin{itemize}
%   \item Our aim: a high level framework for distributed processing of data streams
%   \item Infrastructure abstraction
%   \item Dataflow programming
%   \item Reactive streams
%   \item SGX integration (future work)
%   \item Secure Data Stream: TLS communication? \url{http://curvezmq.org/} (future work)
%   \item Nowadays, streams are everywhere: event streams (Your Mouse Is a Database), IoT, ...
% \end{itemize}

Data streams have taken an inescapable place in communication.
In the world of all-connected and the Internet-of-Things (IoT), on market places, or elsewhere, data is continuously produced and consumed. 
The latest estimates predict a total of \vs{fill} TB/s to be analyzed efificiently and securely within the next 10 years.
This stream processing requires to be reliable, scalable and secure.
This paper introduces \SYS, our initial work on a middleware for developping secure stream processings. 
\SYS supports the implementation, deployment and the execution of stream processing tasks in distributed setting, such as clusters of machines or research testbeds.
It is message-oriented\vs{add ref}, responsive\vs{what do you mean?}, resilient to faults.
Finally it is easy to scale horizontally and vertically\cite{reactivemanifesto}.
%Its design This middleware is designed according to The Reactive Manifesto: it is responsive, resilient, elastic and message driven\cite{reactivemanifesto}.
Briefly, its design is inspired by the dataflow programming paradigm\vs{add ref}: the developer combines together several independent processing components (e.g. mappers, reducers, sinks, shufflers, joiners, etc.) to compose several processing pipes.% an easy way to implement a processing pipe.
The framework integrates an abstraction of the required deployment infrastructure.
It smoothly integrate with industrial-grade lightweight virtualization technologies (e.g. Docker~\cite{}\vs{add ref}).%by deployment automation based on the Docker ecosystem
It aims to be secure: communication channels use the SSL protocol\vs{add ref} for data communication.
Finally, our design intends to exploit the secure processing capabilities offered by trusted hardware \emph{enclaves}, nowdays widely available into mass-market thanks to the introduction of Intel's Software Guard Extensions (SGX)\cite{costan_intel} in the SkyLake processors\vs{add ref}.
% and giving the ability to process datas in trusted enclaves by the integration of the Intel's Software Guard Extension (SGX) hardware.

Few mainstream solutions exist today for distributed stream processing.
Akka Streams Kafka/Reactive Kafka\cite{reactivekafka} allows stream processing on top of Apache Kafka\cite{apachekafka}.
More recently, more open-source solutions like Apache Spark Streaming\cite{apachesparkstreaming}, Apache Storm\cite{apachestorm}, or Infinispan\cite{infinispan} allow developers to quickly setup and deploy a stream processing infrastructure. 
These solutions rely on the Java Virtual Machine (JVM)\cite{}\vs{add ref}.
\SYS proposes an lightweight and low memory-footprint framework that can fully execute inside SGX enclaves, which themselves currently have an hard-limit of 96 MB.
This design choice is to prevent expensive encrypted memory paging mechanisms and serious performance leaks~\cite{brenner_securekeeper:_2016}.  

%But one of our goals is to find out a software technology that can run in a SGX enclave, and the memory of the latter is limited to 96MB for the Intel Skylake processor (the first and only processor including SGX currently)\cite{costan_intel}.
%If the application running inside an enclave needs more memory, an expensive encrypted memory paging mechanism then is used, causing serious performance leaks~\cite{brenner_securekeeper:_2016}.

\vs{recap contributions}
\vs{This paper is organized as follows...bla bla}