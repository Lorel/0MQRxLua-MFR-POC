\section{Introduction}
\label{sec:introduction}

% \begin{itemize}
%   \item Our aim: a high level framework for distributed processing of data streams
%   \item Infrastructure abstraction
%   \item Dataflow programming
%   \item Reactive streams
%   \item SGX integration (future work)
%   \item Secure Data Stream: TLS communication? \url{http://curvezmq.org/} (future work)
%   \item Nowadays, streams are everywhere: event streams (Your Mouse Is a Database), IoT, ...
% \end{itemize}

Data streams are more and more predominant in today's era of big data.
In the world of all-connected and the Internet-of-Things (IoT), on market places, or elsewhere, data is continuously produced and consumed.
%The latest estimates predict a total of \vs{fill} TB/s to be analyzed efficiently and securely within the next 10 years.
This stream processing requires to be reliable, scalable and secure.
This paper introduces \SYS, our initial work on a middleware for developing secure stream processing.
\SYS supports the implementation, deployment and the execution of stream processing tasks in distributed setting, such as clusters of machines or research testbeds.
It is message-oriented~\cite{mom}, responsive, resilient to faults.
Finally it is easy to scale horizontally and vertically.
%Its design This middleware is designed according to The Reactive Manifesto: it is responsive, resilient, elastic and message driven\cite{reactivemanifesto}.
Briefly, its design is inspired by the dataflow programming paradigm\cite{uustalu_essence_2005}: the developer combines together several independent processing components (e.g. mappers, reducers, sinks, shufflers, joiners, etc.) to compose several processing pipes.% an easy way to implement a processing pipe.
The framework integrates an abstraction of the required deployment infrastructure.
It smoothly integrate with industrial-grade lightweight virtualization technologies (e.g. Docker~\cite{docker}).
%by deployment automation based on the Docker ecosystem
It aims to be secure: communication channels use the SSL protocol~\cite{freier2011secure} for data communication.
Finally, our design intends to exploit the secure processing capabilities offered by trusted hardware \emph{enclaves}, nowadays widely available into mass-market thanks to the introduction of Intel's Software Guard Extensions (SGX)\cite{costan_intel} in the SkyLake processors\cite{skylake}.
% and giving the ability to process datas in trusted enclaves by the integration of the Intel's Software Guard Extension (SGX) hardware.

Few mainstream solutions exist today for distributed stream processing.
Reactive Kafka\cite{reactivekafka} allows stream processing on top of Apache Kafka\cite{apachekafka}.
More recently, few open-source solutions (e.g. Apache Spark Streaming\cite{apachesparkstreaming}, Apache Storm\cite{apachestorm}, Infinispan\cite{infinispan}) introduced APIs to allow developers to quickly setup and deploy stream processing infrastructures.
These systems rely on the Java Virtual Machine (JVM)\cite{lindholm2014java}.
However, SGX currently impose a hard memory limit of 128 MB to the enclaved code, at the cost of expensive encrypted memory paging mechanisms and serious performance leaks~\cite{brenner_securekeeper:_2016} when this limit is crossed.
\SYS proposes a lightweight and low memory-footprint framework that can fully execute inside SGX enclaves.
We detail our implementation choices in Section~\ref{sec:implementation}.

%But one of our goals is to find out a software technology that can run in a SGX enclave, and the memory of the latter is limited to 96MB for the Intel Skylake processor (the first and only processor including SGX currently)\cite{costan_intel}.
%If the application running inside an enclave needs more memory, an expensive encrypted memory paging mechanism then is used, causing serious performance leaks~\cite{brenner_securekeeper:_2016}.
To recap, our \textbf{contributions} are the followings: (i) we present the architecture of \SYS, (ii) provide details of our implementation choices, and (iii) present the results of our preliminary evaluation based on a real-world dataset.

This paper is organized as follows.
The architecture of \SYS is described in Section~\ref{sec:architecture}.
Our implementation choices and an example of \SYS program are presented in Section~\ref{sec:implementation}.
Section~\ref{sec:eval} presents our preliminary evaluation with respect to throughput and scalability results.
Before concluding, we wrap up by briefly describing our future work in Section~\ref{sec:future}.
