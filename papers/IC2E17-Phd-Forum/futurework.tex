\section{Future work}
\label{sec:future}


% \begin{itemize}
%   \item Fix bottleneck in routers
%   \item Ability to embed Lua libraries and C or C++ components
%   \item Implements TLS/SSL communication with Curve
%   \item SGX integration
%   \item Architecture automation using Docker API
% \end{itemize}
We plan to extend \SYS along the following directions, with the common goal of making the middleware more secure.
First, the communication between the processing components will rely on secure channels.
%At the moment \SYS is not yet secured.
%To do this, we have to secure both the communication and the processing.
To this end, we plan to integrate \textsc{CurveZMQ}~\cite{zmq:curvezmq}, an open-source authentication and encryption protocol for \zmq.
\textsc{CurveZMQ} is based on a fast, secure elliptic-curve cryptographic primitive provided by the libraries \textsc{CurveCP}\cite{zmq:curvecp} and \textsc{NaCl}\cite{zmq:nacl}.
Moreover, \SYS will generate the authentication certificates in use by the containers on-the-fly at the moment of their deployment.

Secondly, the \SYS will exploit the SGX enclaves to both process and route data.
The computation will be executed inside a trusted enclave, ensuring the encryption of the data and the integrity of the executed code.

We intend to evaluate the impact of these modifications on the performance of the system, in particular with respect to the overall throughput, memory and CPU usage.

The current implementation of \SYS does not provide yet full automation of container deployments.
In particular, we will supersede the current approach based on Docker Compose and instead will rely on the more portable Docker APIs.

Finally, \SYS will allow to embed Lua scripts or native (e.g. C/C++) components inside a node of the processing pipeline.
This feature will improve the usability of the framework.
%Thus, those libraries or components will be callable directly from the code given in an element of the pipeline.
