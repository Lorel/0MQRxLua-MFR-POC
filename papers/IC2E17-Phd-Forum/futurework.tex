\section{Future work}
\label{sec:future}


% \begin{itemize}
%   \item Fix bottleneck in routers
%   \item Ability to embed Lua libraries and C or C++ components
%   \item Implements TLS/SSL communication with Curve
%   \item SGX integration
%   \item Architecture automation using Docker API
% \end{itemize}

At the moment \SYS is not yet secured.
To do this, we have to secure both the communication and the processing.

The communication between the nodes of the processing pipeline will be secure by CurveZMQ\cite{zmq:curvezmq}, an authentication and encryption protocol for \zmq.
This open-source codec is based on a fast, secure elliptic-curve crypto provided by the libraries CurveCP\cite{zmq:curvecp} and NaCl\cite{zmq:nacl}.
\SYS will generate the certificates used by the containers on-the-fly when the pipeline containers are deployed.

In an other hand, the processing of the data inside a worker or its routing inside a worker will be secure by SGX.
The computation will be executed inside a trusted enclave, ensuring the encryption of the data and the integrity of the executed code.


The current implementation of \SYS does not provide yet the automation of container deployments.
The latter is done by a preceding setup executed throw Docker-Compose.
In the future, \SYS will implement it using the Docker API.


Finally, we want to provide all the power of Lua and its extensibility by giving to the developer the ability to embed some Lua libraries or C/C++ components inside a node of the processing pipeline.
Thus, those libraries or components will be callable directly from the code given in an element of the pipeline.
